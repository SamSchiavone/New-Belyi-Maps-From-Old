\documentclass[reqno, 12pt]{amsart}

\usepackage{amssymb}
\usepackage{amsthm}
\usepackage{amsmath}
\usepackage{amsxtra}
\usepackage{mathrsfs}
%\usepackage{enumerate}
\usepackage{comment}
\usepackage{graphicx}
%\usepackage[all]{xy}
\usepackage{placeins}
\usepackage{multicol}
\usepackage{bbm}
\usepackage{cancel}
\usepackage{fullpage}
\usepackage{stmaryrd}
\usepackage{wasysym}
\usepackage{subcaption}
\usepackage{url}
\usepackage{hyperref}

\usepackage[inline, shortlabels]{enumitem}
%\usepackage{calc}
%\usepackage{tikz}
%\usetikzlibrary{arrows,calc,automata,shadows,backgrounds,positioning,intersections,fadings,decorations.pathreplacing,shapes,snakes, matrix}
\usepackage{tikz-cd}

\usepackage{listings}
\usepackage{color}

\definecolor{dkgreen}{rgb}{0,0.6,0}
\definecolor{gray}{rgb}{0.5,0.5,0.5}
\definecolor{mauve}{rgb}{0.58,0,0.82}

\lstset{frame=tb,
  language=Python,
  aboveskip=3mm,
  belowskip=3mm,
  showstringspaces=false,
  columns=flexible,
  basicstyle={\small\ttfamily},
  numbers=none,
  numberstyle=\tiny\color{gray},
  keywordstyle=\color{blue},
  commentstyle=\color{dkgreen},
  stringstyle=\color{mauve},
  breaklines=true,
  breakatwhitespace=true,
  tabsize=3
}

\setlength{\hfuzz}{4pt}

\newtheorem{thm}{Theorem}%[section]
\newtheorem{lem}{Lemma}
\newtheorem{cor}[thm]{Corollary}
\newtheorem{conj}[thm]{Conjecture}
\newtheorem{prop}[thm]{Proposition}
\theoremstyle{definition}  
\newtheorem{defn} [thm] {Definition} 
\newtheorem{claim}[thm]{Claim}
\newtheorem{example} [thm] {Example}
\newtheorem{rem} [thm] {Remark}

\newenvironment{problab}[1]
{\noindent\textbf{Problem #1}.}
{\vskip 6pt}

\theoremstyle{remark}
\newtheorem*{soln}{Solution}

\newenvironment{enumalph}
{\begin{enumerate}\renewcommand{\labelenumi}{\textnormal{(\alph{enumi})}}}
{\end{enumerate}}

\newenvironment{enumroman}
{\begin{enumerate}\renewcommand{\labelenumi}{\textnormal{(\roman{enumi})}}}
{\end{enumerate}}

\newcommand{\C}{\mathbb C}
\newcommand{\D}{\mathbf D}
\renewcommand{\H}{\mathbf H}
\newcommand{\F}{\mathbb F}
\newcommand{\Q}{\mathbb Q}
\newcommand{\R}{\mathbb R}
\newcommand{\Z}{\mathbb Z}
\newcommand{\T}{\mathbb T}
\renewcommand{\P}{\mathbb P}
\renewcommand{\O}{\mathcal O}
\newcommand{\M}{\mathfrak M}
\newcommand{\A}{\mathbb A}
\newcommand{\V}{\mathbb V}
\newcommand{\I}{\mathbb I}
\newcommand{\m}{\mathfrak m}
\renewcommand{\SS}{\mathbb S}

\newcommand{\Cscr}{\mathscr C}
\newcommand{\Dscr}{\mathscr D}
\newcommand{\Fscr}{\mathscr F}
\newcommand{\Gscr}{\mathscr G}

\renewcommand{\L}{\mathcal{L}}

\renewcommand{\Re}{\text{Re}}
\renewcommand{\Im}{\text{Im}}

\newcommand{\gl}{\mathfrak gl}
\renewcommand{\sl}{\mathfrak sl}
\newcommand{\s}{\mathfrak s}
\renewcommand{\a}{\mathfrak a}
\newcommand{\p}{\mathfrak p}
\newcommand{\q}{\mathfrak q}
\newcommand{\normal}{\trianglelefteq}

\newcommand{\Cov}{\text{Cov}}
\newcommand{\charac}{\text{char}}
\newcommand{\la}{\langle}
\newcommand{\ra}{\rangle}
\newcommand{\wt}{\widetilde}

\newcommand{\Auniv}{{A_\text{univ}}}
\newcommand{\Runiv}{{R_\text{univ}}}
\DeclareMathOperator{\gdeg}{gdeg}

\newcommand{\Kbar}{\overline{K}}

\renewcommand{\div}{\text{div}}

\DeclareMathOperator{\lcm}{lcm}
\DeclareMathOperator{\img}{img}
\DeclareMathOperator{\ann}{ann}
\DeclareMathOperator{\Tor}{Tor}
\DeclareMathOperator{\tr}{tr}
\DeclareMathOperator{\Hom}{Hom}
\DeclareMathOperator{\Mor}{Mor}
\DeclareMathOperator{\End}{End}
\DeclareMathOperator{\Aut}{Aut}
\DeclareMathOperator{\Gal}{Gal}
\DeclareMathOperator{\Spl}{Spl}
\DeclareMathOperator{\sgn}{sgn}
\DeclareMathOperator{\ord}{ord}
\DeclareMathOperator{\Ob}{Ob}
\DeclareMathOperator{\genus}{genus}
\DeclareMathOperator{\ad}{ad}
\DeclareMathOperator{\GL}{GL}
\DeclareMathOperator{\SL}{SL}
\DeclareMathOperator{\SO}{SO}
\DeclareMathOperator{\Sp}{Sp}
\DeclareMathOperator{\Spec}{Spec}
\DeclareMathOperator{\Stab}{Stab}
\DeclareMathOperator{\Frac}{Frac}
\DeclareMathOperator{\Div}{Div}
\DeclareMathOperator{\Pic}{Pic}
\DeclareMathOperator{\supp}{supp}
\DeclareMathOperator{\wasy}{\wasylozenge}
\DeclareMathOperator{\Frob}{Frob}
\DeclareMathOperator{\Nm}{Nm}
\DeclareMathOperator{\rank}{rank}

\DeclareMathOperator{\trd}{trd}
\DeclareMathOperator{\nrd}{nrd}

\DeclareMathOperator{\Supp}{Supp}
\DeclareMathOperator{\res}{res}

\DeclareMathOperator{\id}{id}

\newcommand{\trouble}[1]{{\color{green} [#1]}}

%\newcommand{\res}[3]{\left(\frac{#1}{#2} \right)_{#3}}

\renewcommand{\thesection}{\Roman{section}} 

\usepackage{mathpple}
%\usepackage{pxfonts}

\everymath{\displaystyle}
\tikzset{commutative diagrams/.cd, arrow style = tikz, diagrams = {>=latex}}


\begin{document}

\title{Belyi stuff}
\author{Sam Schiavone}
%\date{30 June 2017}
\date{\today}

\maketitle

\section{Bielliptic Belyi}

Combining my recent interests of Belyi maps and glueing curves along their $2$-torsion, I decided to look at which genus $2$ Belyi maps we've computed so far are defined on bielliptic curves, i.e., genus $2$ hyperelliptic curves $X$ with distinct morphisms $\psi_1: X \to E_1$ and $\psi_2: X \to E_2$ where $E_1$ and $E_2$ are elliptic curves. You can easily spot at least some of these from their equations: when $y^2 = f(x)$ and $f$ contains only even powers of $x$. Let's see why in an example: consider
\begin{align*}
X: y^{2} = x^{6} + 4 x^{4} + 6 x^{2} + 3 \, ,
\end{align*}
the curve from \texttt{6T6-6\_6\_3.3-a}. Replacing $x^2$ by $x$, we obtain the elliptic curve
$$
E_1: y^2 = x^3 + 4 x^2 + 6 x + 3 \, .
$$
Homogenizing the equation defining $X$ (gradedly, of course), we have
$$
y^{2} = x^{6} + 4 x^{4} z^2  + 6 x^{2} z^4 + 3 z^6
$$
and dehomogenizing by setting $x = 1$, we have
$$
y^{2} = 1 + 4 z^2  + 6 z^4 + 3 z^6
$$
find the second elliptic curve
$$
E_2: y^2 = 1 + 4 z + 6 z^2 + 3 z^3 \, .
$$
So we get the maps
\begin{align*}
X &\to E_1\\
(x,y) &\mapsto (x^2, y)
\end{align*}
and
\begin{align*}
X &\to E_2\\
(z,y) &\mapsto (z^2, y) \, .
\end{align*}

on the open subsets where these equations hold. (I could work out the maps on the homogenized version, but I'm too lazy.)

Anyway, here's a list of all the genus $2$ bielliptic Belyi maps I've found so far.
\begin{itemize}
\item
\url{https://beta.lmfdb.org/Belyi/5T4/5/5/5/a}

\item
\url{https://beta.lmfdb.org/Belyi/6T1/6/6/3.3/a}

\item
\url{https://beta.lmfdb.org/Belyi/6T5/6/6/3.3/a}

\item
\url{https://beta.lmfdb.org/Belyi/6T6/6/6/3.3/a}

\item
\url{https://beta.lmfdb.org/Belyi/7T7/7/4.3/4.3/b}
\end{itemize}

I was hoping to find an example where we have computed a Belyi map on the genus $2$ curve and both elliptic curves. Maybe we could find a commutative diagram?
$$
\begin{tikzcd}
{} & X \ar[swap]{dl}{\psi_1} \ar[]{dr}{\psi_2} \ar[]{dd}{\varphi}& {}\\
E_1 \ar[swap]{dr}{\varphi_1} & {} & E_2 \ar[]{dl}{\varphi_2}\\
{} & \P^1 & {}
\end{tikzcd}
$$

I came close, but no cigar so far. Here's the near-example. The map \texttt{6T6-6\_6\_3.3-a} (\url{https://beta.lmfdb.org/Belyi/6T6/6/6/3.3/a}) is defined on the curve \texttt{1728.b.442368.1} (\url{https://beta.lmfdb.org/Genus2Curve/Q/1728/b/442368/1}). It is a $2$-glueing of the elliptic curves \texttt{36.a4} (\url{https://beta.lmfdb.org/EllipticCurve/Q/36/a/4}) and \texttt{48.a5} (\url{https://beta.lmfdb.org/EllipticCurve/Q/48/a/5}). The curve \texttt{36.a4} has minimal Weierstrass model
$$
E_1: y^2 = x^3 + 1
$$
and we've computed a couple Belyi maps on $E_1$, namely the Euclidean map \texttt{3T1-3\_3\_3-a} (\url{https://beta.lmfdb.org/Belyi/3T1/3/3/3/a}) and the hyperbolic map \texttt{9T20-9\_2.2.2.1.1.1\_6.3-a} (\url{https://beta.lmfdb.org/Belyi/9T20/9/2.2.2.1.1.1/6.3/a}). But despite some near misses, we haven't gotten to a Belyi map on \texttt{48.a5}. However, while searching for Belyi maps of conductor $48$, I did make an interesting observation.

\section{Curve with many maps}

I noticed that the elliptic curve with label \texttt{48.a6}
\begin{align*}
E: y^2 &= x^3 + 47/768 x + 2359/55296
\end{align*}
admits a bunch of Belyi maps of low degree. First we have one of my favorites, the genus 1 hyperbolic triple of minimal degree \texttt{4T5-4\_4\_3.1-a} (\url{https://beta.lmfdb.org/Belyi/4T5/4/4/3.1/a}). Today I noticed there are a bunch of degree 8 maps defined on the isomorphic curve
$$
E': y^2 = x^3 + 47/12288 x + 2359/3538944
$$
(this equation just differs by a factor of $2$), namely \texttt{8T46-8\_8\_3.1.1.1.1.1-a} (\url{https://beta.lmfdb.org/Belyi/8T46/8/8/3.1.1.1.1.1/a}), \texttt{8T47-8\_2.2.2.2\_4.3.1-a} (\url{https://beta.lmfdb.org/Belyi/8T47/8/2.2.2.2/4.3.1/a}), and \texttt{8T47-8\_6.2\_4.1.1.1.1-a} (\url{https://beta.lmfdb.org/Belyi/8T47/8/6.2/4.1.1.1.1/a}). Even curiouser, all 3 have nearly the same equation! I wonder if there's some explanation for this. Maybe they arise from composing the degree $4$ map with a degree $2$ map, or maybe from taking a fiber product?

\section{Different fields of definition for curve and map}

When I was first computing genus $1$ Belyi maps, I ran into a problem. I thought that the Belyi map $\varphi: X \to \P^1$ with label \texttt{5T3-5\_4.1\_4.1-a} was defined over $\Q$, as the curve $X$ was. But it turned out that $\varphi$ is only defined over $\Q(i)$: see \url{https://beta.lmfdb.org/Belyi/5T3/5/4.1/4.1/a}. I saw another instance of this today: \url{https://beta.lmfdb.org/Belyi/7T7/7/4.3/4.3/c}.

%\nocite{*} % use this to have all references listed, not just those cited
%\bibliographystyle{amsplain}
\bibliographystyle{alpha}
%\addcontentsline{toc}{chapter}{References}
%\bibliography{references}

\end{document} 
